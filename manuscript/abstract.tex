\begin{abstract}

	Oxygenation of tumours and the effect of hypoxia in cancer cell metabolism is a
	widely studied subject.
	\acrfull{hif}, the main actor in the cell response to hypoxia,
	represents a potential target in cancer therapy.
	\acrshort{hif} is involved in many biological processes such as cell
	proliferation, survival, apoptosis, angiogenesis, iron metabolism and glucose
	metabolism.
	This protein regulates the expressions of \acrfull{ldh} and \acrfull{pdh}, both
	essential for the conversion of pyruvate to be used in aerobic and anaerobic
	pathways.
	\acrshort{hif} upregulates \acrshort{ldh}, increasing the conversion of
	pyruvate into lactate which leads to higher secretion of lactic acid by the cell
	and reduced pH in the microenvironment. \acrshort{hif} indirectly downregulates
	\acrshort{pdh}, decreasing the conversion of pyruvate into Acetyl Coenzyme A
	which leads to reduced usage of the \acrfull{tca} cycle in aerobic pathways.
	Upregulation of \acrshort{hif} may promote the use of anaerobic pathways for
	energy production even in normal extracellular oxygen conditions.
	Higher use of glycolysis even in normal oxygen conditions is called the Warburg
	effect.
	In this paper, we focus on \acrshort{hif} variations during tumour growth and
	study, through a mathematical model, its impact on the two metabolic key genes
	\acrshort{pdh} and \acrshort{ldh}, to investigate its role in the emergence of
	the Warburg effect.
	Mathematical equations describing the enzymes regulation pathways were solved
	for each cell of the tumour represented in an agent-based model to best capture
	the spatio-temporal oxygen variations during tumour development caused by cell
	consumption and reduced diffusion inside the tumour.
	Simulation results show that reduced \acrshort{hif} degradation in normoxia can
	induce higher lactic acid production.
	The emergence of the Warburg effect appears after the first period of hypoxia
	before oxygen conditions return to a normal level.
	The results also show that targeting the upregulation of \acrshort{ldh} and the
	downregulation of \acrshort{pdh} could be relevant in therapy.
	
\end{abstract}
