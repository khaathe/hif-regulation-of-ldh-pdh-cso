\section*{Conclusion}
\addcontentsline{toc}{section}{Conclusion}

The main interest of the model is its ability to qualitatively describe \acrshort{hif} expression in tumour development over time with oxygen diffusion that depends on both cell consumption and cell density in the tumour to obtain a more realistic diffusion.
Results of the model show that varying oxygen levels and reduced \acrshort{hif} degradation can cause increased glycolytic activity in normal oxygen levels.
In the model, the emergence of the Warburg effect is preceded by a first period of hypoxia before returning to normoxic concentrations.
This suggests that adaptation to environmental conditions is the primary phenomenon to understanding the Warburg effect.
Interfering with the genetic activity of \acrshort{hif} or its effect on \acrshort{ldh} and \acrshort{pdh} may be used in therapy to induce specific behaviour in the cell. 
