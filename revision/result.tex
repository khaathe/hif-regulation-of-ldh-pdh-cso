\section{Results}

\subsection{Qualitative exploration of the model at the cell scale}

A well-known phenomenon is the Warburg effect, increased production of lactic acid by the tumour \cite{Warburg1927} even in normoxia \cite{Smallbone2007, Leung2017, Robey2005}.
A qualitative study of the genetic deregulations at the cell scale would reveal how it impacts lactic-acid production to investigate the appearance of the Warburg effect.
The primary aim of this study is to investigate the role of genetic regulations in cell metabolic changes.

In our mathematical model, the regulating effect of a gene on another is mainly driven by the $\gamma$ parameter in the shifted-Hill function.
Setting this parameter equal to 1 simulate a loss of the regulating function.
An over-sensitivity of a gene by its regulator is modelled by setting the $\gamma$ parameter to 40, the maximum defined in the model from \cite{Li2020}.
Results of a few regulations are shown in figure \ref{fig:acid-prod-map}. 

When no genetic deregulations are applied to the model (figure \ref{fig:acid-prod-map}.A), protons production range from 0.0001 mmol/L/min to 0.001 mmol/L/min with normal $\gamma$ parameters.
Around 0.01 mmol/L oxygen (1\%), the cell progressively increases its \protons secretion rate from 0.0001 mmol/L/min to the maximum 0.001 mmol/L/min.

In our model, when \acrshort{hif} is not subjected to oxygen degradation (figure \ref{fig:acid-prod-map}.B), the rate of \protons production is only influenced by glucose concentration.
In this case, cell's lactic-acid secretion rate can reach 0.001 mmol/L/min even in normal oxygen pressure, as a result of the Warburg effect.
Increased degradation of \acrshort{hif} in oxygen (figure \ref{fig:acid-prod-map}.C) reduces the oxygen threshold at which the cell has a lactic-acid secretion rate of 0.001 mmol/L/min.
Lower levels of oxygen are needed to reach the maximal secretion rate compared to the normal degradation rate of \acrshort{hif}.
With no deregulation (figure \ref{fig:acid-prod-map}.A), the lactic-acid secretion rate starts to increase at around 0.019 mmol/L of oxygen and reach a maximum at around 0.08 mmol/L. 
With increased \acrshort{hif} degradation by oxygen (figure \ref{fig:acid-prod-map}.C), this span is reduced and lactic-acid secretion increases at around 0.012 mmol/L of oxygen. 
Similar to our result, a model from \cite{Jia2019} shows that a lower degradation rate of \acrshort{hif} increases the chance that cells use glycolysis instead of \acrshort{oxphos}, which will increase lactic acid secretions.

Inhibiting \acrshort{ldh} sensitivity to \acrshort{hif} (figure \ref{fig:acid-prod-map}.D) causes the maximum lactic-acid secretion rate to fall to 0.0008 mmol/L/min. 
Increasing \acrshort{ldh} sensitivity to \acrshort{hif} does not permit the cell to have a higher \protons production rate in normoxia, while a decrease prevents a high \protons production rate in hypoxia (results not shown). 

Interfering with \acrshort{pdk} sensitivity to \acrshort{hif} or \acrshort{pdh} sensitivity 
to \acrshort{pdk} seems to have no effect on acid production in the model but on oxygen 
consumption by the cell (results not shown).

\begin{figure}
    \centering
    \includegraphics[width=\textwidth]{img/acid-prod-map-simplified}
    \caption{ 
        Influence of genetic upregulation or inhibition on the production rate of protons 
        at different glucose and oxygen concentrations.
        (A) Result with no genetic deregulation.
        ($\gamma_{O_2 \to u} = 10.0$, $\gamma_{u \to v} = 3.61$, $\gamma_{u \to w} = 6.97$, $\gamma_{w \to z} = 0.14$)
        (B) Result with inhibition of the oxygen-dependant degradation of \acrshort{hif}.
        ($\gamma_{O_2 \to u} = 1.0$, $\gamma_{u \to v} = 3.61$, $\gamma_{u \to w} = 6.97$, $\gamma_{w \to z} = 0.14$)
        (C) Result with over-degradation of \acrshort{hif} by oxygen.
        ($\gamma_{O_2 \to u} = 40.0$, $\gamma_{u \to v} = 3.61$, $\gamma_{u \to w} = 6.97$, $\gamma_{w \to z} = 0.14$)
        (D) Result with loss of upregulation of \acrshort{ldh} by \acrshort{hif}.
        ($\gamma_{O_2 \to u} = 10.0$, $\gamma_{u \to v} = 3.0$, $\gamma_{u \to w} = 6.97$, $\gamma_{w \to z} = 0.14$)
    }
    \label{fig:acid-prod-map}
\end{figure}

\subsection{Exploration of environment and genetic  properties on the emergence of the Warburg phenotype}

\subsubsection{Influence of environmental oxygen conditions}

The Warburg effect is currently defined as high production of acidity due to the use of glycolysis even in normoxia \cite{Jia2019,  Courtnay2015, Damaghi2020}.
We ran several simulations with different environmental oxygen conditions to assess whether microenvironmental conditions only, can induce a Warburg effect in the model.


\begin{figure}
    \centering
    \includegraphics[width=\textwidth]{img/environmental-condition/tumour_current_phase}
    \caption{ 
        Evolution of tumour growth at different times in different conditions.
        In oscillating conditions, the oxygen concentration is slowly decreased from normoxia to hypoxia during 6 hours, then cells are slowly put back in normoxia at the same rate.
        This process is repeated until the end of the simulation.
    }
    \label{fig:tumour-state-environment}
\end{figure}    

Figure \ref{fig:tumour-state-environment} shows how oxygen conditions affect tumour growth.
In oscillating conditions, the oxygen concentration varies between physiological normoxia and pathological hypoxia and reverse every 6 hours until the end of the simulation.
Kinetics of \acrshort{hif} show a peak after 6h and a decrease to an equilibrium state after  24h-48h.
We choose to simulate 6h-period of hypoxia/normoxia to avoid the cell reaching an equilibrium and to simulate stressful conditions with a high response to a low level of oxygen.
Constant hypoxia slows down tumour growth and reduces tumour diameter compared to normoxia.
In all 3 different conditions, the centre of the tumour is composed of dead cells surrounded by living cells at the periphery.
Only in normoxia and varying oxygen conditions, some cells in the centre of the tumour do continue to divide (only visible after 7 days of growth).
This may be due to the changes in the tumour microenvironment with the increased cell death at the centre.
As the cells die, more nutrients will be available to quiescent cells to enable them to renter proliferating phase.
Moreover, spatial changes due to the shrinkage of dead cells can influence the availability of nutrients at the centre.
This might show a mechanism by which the tumour can grow back after a period of harsh conditions, for example quiescence can be a mechanism to avoid drugs effect for the tumour cell \cite{Alarcon2011}. 
Necrotic core has been observed in biological experiments run in the lab (figure \ref{fig:spheroid-experiment}).

\begin{figure}
    \centering
    \includegraphics[width=0.5\textwidth]{img/U87-GFP}
    \caption{
        Picture of a spheroid grown for 30 days in an experiment run in the laboratory.
        Cells were marked using the fluorescent proteins Green FLuorescent Protein (GFP) and Sulforhodamine B (SRB).
        Living cells are colored in green, dying cells appear in red.
        The centre of the tumour is composed of hypoxic and dead cells, both do not emit fluorescence.
    }
    \label{fig:spheroid-experiment}
\end{figure}

It seems that varying the concentration of oxygen from normoxia to hypoxia, and reversing this process, every 6 hours doesn't affect the diameter of the tumour at the end of the simulation.
However, a ring of necrotic cells in the swelling phase appears thicker than in other conditions.

\begin{figure}
    \centering
    \includegraphics[width=\textwidth]{img/environmental-condition/acid_rate_o_extra}
    \caption{ 
        Acid production rate following oxygen extracellular concentrations at different times in different conditions. 
        The red line indicates the hypoxia threshold.
        In oscillating conditions, oxygen concentration is slowly decreased from normoxia to hypoxia during 6h, then oxygen is increased to normoxia at the same rate.
        This process is repeated until the end of the simulation.
        Only living cells are represented on the graph.
        The green rectangle represents the region corresponding to a Warburg effect.
    }
    \label{fig:prod-rate-environment}
\end{figure}

Results in figure \ref{fig:prod-rate-environment} show acid production according to the extracellular oxygen concentration. 
Red line y-axis intercept is equal to 0.02085 mmol/L (2\% O\textsubscript{2}), which corresponds to the threshold of hypoxia in physiological conditions.
It is the level at which \acrshort{hif} has a half-maximal response as well \cite{McKeown2014}.
Cells above this level are considered to be in normoxia while the rest of the cells are in hypoxia.
Levels of extracellular oxygen fall below the hypoxia threshold after 2 days of growth in normoxic conditions (a necrotic core in the centre of the tumour has already formed).
Due to poor oxygen concentration, cells with higher glycolytic activity appear and reach a \protons production rate of almost $5 \times 10^{-4} \text{mmol/L/min}$.
The maximum glycolytic activity of cells falls at 7 days of growth because of reduced glucose availability.
When tumour growth is started in hypoxic conditions, high glycolytic activity is present after only one day of growth.
In those conditions, the way the cell produces its energy is influenced only by glucose concentrations (similar to the result shown in figure \ref{fig:acid-prod-map}).
Therefore, hypoxic conditions directly select cells with high glycolytic activity. 

\begin{figure}
    \centering
    \includegraphics[width=\textwidth]{img/environmental-condition/ldh_and_pdh}
    \caption{ 
        Plot of level of \acrshort{pdh} against the level of \acrshort{ldh} coloured by the extracellular oxygen concentration.
        The graph shows the results at different times for different conditions.
        In oscillating conditions, oxygen concentration is slowly decreased from normoxia to hypoxia during 6h, then oxygen is increased to normoxia at the same rate.
        This process is repeated until the end of the simulation.
        Only living cell are represented on the graph.
    }
    \label{fig:ldh-pdh-environment}
\end{figure}    

The fact that, in the model, hypoxia may select cells with high glycolytic activity is supported by the levels of \acrshort{ldh}/\acrshort{pdh} genes presented in figure \ref{fig:ldh-pdh-environment}.
In normoxia, cells have a level of \acrshort{ldh} and \acrshort{pdh} of 1 for both, it can be associated to an oxidative state.
In hypoxia, \acrshort{ldh} level reaches 3.0 and \acrshort{pdh} level falls to 0.25, it can be associated to a glycolytic state.
At the beginning of the simulation in normoxic conditions, cells have 1:1 \acrshort{ldh}/\acrshort{pdh} levels.
As the simulation goes, oxygen becomes less available.
Thus \acrshort{ldh} level increases while \acrshort{pdh} level decreases.
The result in normoxic conditions shows that cells migrate from an oxidative to a glycolytic state as oxygen concentration decreases.
Cells around 2:0.5 \acrshort{ldh}/\acrshort{pdh} levels have a hybrid state where they rely on both nutrients to produce \acrshort{atp}.
Again hypoxia selects for cells with high levels of \acrshort{ldh} and low levels of \acrshort{pdh}, suppressing the possibility for the cell to adopt a hybrid state.

Interestingly, extracellular oxygen concentration after 7 days is higher when oxygen varies between normoxia and hypoxia every 6 hours than in constant normoxia.
Since cells are put in hypoxia several times a day, they rely more on glycolysis and consume less oxygen.
Cells with higher glycolytic activity ($2.5 \times 10^{-4} \text{mmol/L/min}$) even above the threshold of hypoxia appear at 2 days.
It suggests that the Warburg Effect can be caused by environmental conditions with rapid variations.
Combined with figure \ref{fig:ldh-pdh-environment}, genetic levels seems to indicate that cells cannot enter a complete oxidative state and are trapped either in a hybrid or a glycolytic state. 

\subsubsection{Influence of the intrinsic genetic properties of the cell}

\begin{figure}
    \centering
    \includegraphics[width=\textwidth]{img/genetic-perturbation/tumour_current_phase}
    \caption{
        Evolution of tumour growth at different time with different genetic perturbations: reduced oxygen induced degradation  of \acrshort{hif} ($\gamma O_2 \to \acrshort{hif} = 8.0 $), lower use of glycolysis in hypoxic conditions ( $\gamma \acrshort{hif} \to \acrshort{ldh} = 2.0 $) and lower effect of hypoxia on oxygen consumption ( $\gamma \acrshort{pdk} \to \acrshort{pdh} = 0.7 $).
        The normoxia conditions represent the growth with base parameters or no genetic deregulation ($\gamma O_2 \to \acrshort{hif} = 10.0 $, $\gamma \acrshort{hif} \to \acrshort{ldh} = 3.61 $, $\gamma \acrshort{pdk} \to \acrshort{pdh} = 0.14 $). 
        New values have been selected following a qualitative exploration of the parameters effect on the model at the cell scale.
        Tumour growth was initiated in normoxia in all the simulations.
    }
    \label{fig:tumour-state-genetic}
\end{figure}

In figure \ref{fig:tumour-state-genetic}, tumour growth is initiated in normoxia. 
Extracellular oxygen concentrations only vary due to cells consumption and reduced diffusion in the tumour.
Only genetics regulations have been modified between each simulation to assess the impact of different genetic deregulations (mutations or epigenetic alterations) on tumour growth and cell metabolism. 
Results are similar to normoxic conditions with no genetic deregulations (figure \ref{fig:tumour-state-environment}). 
When reducing inhibition of \acrshort{pdh} by \acrshort{pdk}, tumour radius at 7 days of growth is lower than in normoxia and higher than in hypoxia with no mutation. 

\begin{figure}
    \centering
    \includegraphics[width=\textwidth]{img/genetic-perturbation/acid_rate_o_extra}
    \caption{
        Acid production rate following oxygen extracellular concentrations at different times with different genetic perturbations.
        The red line indicates the hypoxia threshold.
        Only living cell are represented on the graph.
        Three genetics pertubations have been selected: reduced oxygen induced degradation  of \acrshort{hif} ($\gamma O_2 \to \acrshort{hif} = 8.0 $), lower use of glycolysis in hypoxic conditions ( $\gamma \acrshort{hif} \to \acrshort{ldh} = 2.0 $) and lower effect of hypoxia on oxygen consumption ( $\gamma \acrshort{pdk} \to \acrshort{pdh} = 0.7 $).
        Tumour growth was initiated in normoxia.
        The green rectangle represents the region corresponding to a Warburg effect.
    }
    \label{fig:prod-rate-genetic}
\end{figure}

Figure \ref{fig:prod-rate-genetic} shows that cells start to become hypoxic after day 1, reaching a majority by day 2.
After 7 days with a reduced \acrshort{hif} degradation rate by oxygen, extracellular oxygen goes back to normoxic levels yet cells have a higher acid production rate that corresponds to a Warburg effect.
In this case, we suppose that cells slowly drain oxygen levels in the environment to a point where hypoxia is reached.
Due to poor oxygen conditions, cells adapt their metabolism to enter a glycolytic state that they keep even if the oxygen supply goes back above 2 \%O\textsubscript{2}.
Together with the result in figure \ref{fig:ldh-pdh-genetic}, this might be caused by a delay in the response from returning to normal conditions since \acrshort{hif} regulation by O\textsubscript{2} is affected.
While some cells have levels of \acrshort{ldh} greater than 2 and \acrshort{pdh} lower than 0.50 (hybrid to glycolytic state), some have a ratio of \acrshort{ldh}/\acrshort{pdh} almost equal to 1:1.
This suggests that the Warburg Effect isn't irreversible with a reduced \acrshort{hif} degradation rate by oxygen alone. 

\begin{figure}
    \centering
    \includegraphics[width=\textwidth]{img/genetic-perturbation/ldh_and_pdh}
    \caption{
        Plot of level of \acrshort{pdh} against the level of \acrshort{ldh} coloured by the extracellular oxygen concentration.
        The graph shows the results at different times with three different genetic perturbations: reduced oxygen induced degradation  of \acrshort{hif} ($\gamma O_2 \to \acrshort{hif} = 8.0 $), lower use of glycolysis in hypoxic conditions ( $\gamma \acrshort{hif} \to \acrshort{ldh} = 2.0 $) and lower effect of hypoxia on oxygen consumption ( $\gamma \acrshort{pdk} \to \acrshort{pdh} = 0.7 $).
        Tumour growth was initiated in normoxia.
    }
    \label{fig:ldh-pdh-genetic}
\end{figure}

As expected, reducing the increase in \acrshort{ldh} levels due to \acrshort{hif} response doesn't induce a high acidification rate in normoxia but affects the maximum acid production rate and level of \acrshort{ldh}.
Instead of inducing a glycolytic phenotype, it seems to repress it.

Reducing the inhibiting power of \acrshort{pdk} on \acrshort{pdh} allows the cell to keep a higher \acrshort{pdh} level, a key enzyme for oxygen consumption and oxidative state in the model.
Cells exhibit an acid production rate similar to those in hypoxic conditions after 2 and 7 days, compared to other genetics deregulation.
While in normoxia with no genetic deregulation cells seem to fluctuate around the threshold of hypoxia, here they are all below this level.
Since \acrshort{pdh} isn't effectively regulated by \acrshort{hif}, the cell tends to stay in an oxidative state and rely less on glycolysis.
We can suppose that cells consume oxygen even when the level fall, creating further harder conditions.
Results also show that adaptation to hypoxia is delayed and the cell only adopts a glycolytic state at oxygen conditions near-pathological hypoxia.
\acrshort{pdh} levels don't fall far below 0.75 even after 7 days of growth compared to others conditions, indicating that cells can only adopt an oxidative or hybrid state.
