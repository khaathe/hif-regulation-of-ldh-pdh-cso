\DIFaddbegin \appendix

\section{\DIFadd{Appendix}}

\subsection{\DIFadd{Model from Robertson-Tessi and al. }}
\DIFadd{In this section, we will briefly describe the model from }\cite*{Robertson-Tessi2015}\DIFadd{.
The author describe an }\acrfull{hca} \DIFadd{model which is simulated on a grid with square lattice.
Each grid point can be empty or occupied by a cell.
Cell consumptions, energy production and acid secretion are computed by several functions and included into a PDE to determine substrates concentrations accross the microenvironment.
}

\DIFadd{Cells will produce a target level of }\acrshort{atp} \DIFadd{using mostly aerobic pathways when oxygen is available. 
Oxygen consumption is described using a Michaelis-Menten function.
}\begin{equation}
    \DIFadd{\label{appendix:fo-robertson}
    f_O = - V_O }{ \DIFadd{\frac{O}{ O + k_O} }}
\DIFadd{}\end{equation}
\DIFadd{with $V_O$ the maximum oxygen consumption rate, $O$ the extracellular oxygen concentration and $k_O$ the extracellular oxygen concentration at which oxygen consumption is half maximum. }\\
\DIFadd{Anaerobic pathways are an alternative way to produce }\acrshort{atp} \DIFadd{when cell requirement is not met using aerobic pathways.
Therefore, glucose consumption is increased to make up for the lack of }\acrshort{atp}\DIFadd{.
Glucose consumption is determined by a Michaelis-Menten equation with a modified term to reflect the increase of glucose consumption when oxygen is not available.
}\begin{equation}
    \DIFadd{\label{appendix:fg-robertson}
    f_G = - ( \frac{p_{G} A_0}{2} + \frac{27 f_O}{10} ) }{ \DIFadd{\frac{G}{ G + k_G} }}
\DIFadd{}\end{equation}
\DIFadd{where $G$ is the extracellular glucose concentration, $k_G$ the half-maximum, $A_0$ the target level of }\acrshort{atp} \DIFadd{and $p_G$ is a coefficient to account for the increased glucose metabolism observed for many tumour cells.
For normal cells have $p_G = 1$, while for tumour cells $p_G>1$.
With higher values of $p_G$, cells consume more glucose to produce }\acrshort{atp} \DIFadd{than normal cells.
Effective }\acrshort{atp} \DIFadd{production is deduced from nutrient consumption.
}\begin{equation}
    \DIFadd{\label{appendix:fa-robertson}
    f_{A} = - ( 2 f_G + \frac{27 f_O}{5} ) }\\
\DIFadd{}\end{equation}
\DIFadd{The amount of glucose used through anaerobic pathways determine the quantity of protons secreted.
}\begin{equation}
    \DIFadd{\label{appendix:fh-robertson}
    f_{H^+} = k_H ( \frac{ 29 ( p_G V_O + f_O) }{5})
}\end{equation}
\DIFadd{where $k_H$ is a proton buffering coefficient.
}

\DIFadd{Cell consumptions and productions are used to dermine the concentration of a molecule accross the microenvironment.
The concentration $C$ of a molecule $x$ is described by the equation :
}\begin{equation}
    \DIFadd{\label{appendix:pde-robertson}
    \frac{\partial C}{\partial t} = D \nabla^2 C + f(C,p) 
}\end{equation}
\DIFadd{where $D$ is a diffusion coefficient, $f$ is a function accounting for the impact of cell with parameters $p$ on the local substrate concentration $C$.
Diffusing substrates are oxygen ($O$), glucose ($G$) and protons ($H$).
Functions $f_O$, $f_G$ and $f_H$ are used for the $f$ term in equation }\ref*{appendix:pde-robertson}\DIFadd{, $f_A$ is used to drive the cells behaviour in the cellular automaton. }\DIFaddend